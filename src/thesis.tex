% arara: pdflatex
% arara: pdflatex
% arara: pdflatex

% options:
% thesis=B bachelor's thesis
% thesis=M master's thesis
% czech thesis in Czech language
% slovak thesis in Slovak language
% english thesis in English language
% hidelinks remove colour boxes around hyperlinks

\documentclass[thesis=M,czech]{FITthesis}[2019/12/23]

\usepackage[utf8]{inputenc} % LaTeX source encoded as UTF-8

% \usepackage{amsmath} %advanced maths
% \usepackage{amssymb} %additional math symbols

\usepackage{dirtree} %directory tree visualisation
\usepackage{svg}
\usepackage{float}




\usepackage{amsthm}

\theoremstyle{plain}
\newtheorem{thm}{Definice}[chapter] % reset theorem numbering for each chapter

\theoremstyle{definition}
\newtheorem{defn}[thm]{Definice} % definition numbers are dependent on theorem numbers
\newtheorem{exmp}[thm]{Příklad} % same for example numbers

% % list of acronyms
% \usepackage[acronym,nonumberlist,toc,numberedsection=autolabel]{glossaries}
% \iflanguage{czech}{\renewcommand*{\acronymname}{Seznam pou{\v z}it{\' y}ch zkratek}}{}
% \makeglossaries

\newcommand{\tg}{\mathop{\mathrm{tg}}} %cesky tangens
\newcommand{\cotg}{\mathop{\mathrm{cotg}}} %cesky cotangens

% % % % % % % % % % % % % % % % % % % % % % % % % % % % % % 
% ODTUD DAL VSE ZMENTE
% % % % % % % % % % % % % % % % % % % % % % % % % % % % % % 

\department{Katedra počítačových systémů}
\title{Multimodální navigace a její nasazení v škálovatelné architektuře}
\authorGN{Jan} %(křestní) jméno (jména) autora
\authorFN{Sokol} %příjmení autora
\authorWithDegrees{Bc. Jan Sokol} %jméno autora včetně současných akademických titulů
\author{Jan Sokol} %jméno autora bez akademických titulů
\supervisor{Ing. Ondřej Guth Ph.D.}
\acknowledgements{TODO.}
\abstractCS{Tato práce se zabývá návrhem plánovače cest v  geoprostorových grafech, konkrétně s omezením na cesty ve městě. Plánovač nabízí různé způsoby dopravy a v určitých kombinacích také využívá vícera možností prostředků najednou. Plánovací služba je navržena dle principů tzv. mikroslužeb  (microservices). K plánovací službě je přístup navržen pomocí REST API. Do souvislosti s mikroslužbami je často spojováno využití technologií jako Docker a Kubernetes, kterých bude využito k nasazení aplikace do distribuovaného a škálovaného systému v druhé části práce. Při nasazení do Kubernetes je brán důraz na zabezpečení aplikace, bude tedy popsána autentikace a autorizace přístupů do systému. Část práce je věnována možnostem škálování aplikace v distribuovaném systému Kubernetes, s řešením přivýpadcích serverů či datacenter. Bude brán důraz na vysokou dostupnost aplikace jak při běžném provozu, tak při častém nasazování.}
\abstractEN{This thesis deals with design of a trip planner in geospatial graphs, with a limitation of routes in cities. Route planner offers various means of transport. Multiple ways of transport are combined into one single trip when certain combinations are used. Route planning service is designed using principles of so called microservices. Access to the planning results is designed using REST API. Technologies like Docker and Kubernetes are usually connected with microservices, those technologies will be used to deploy planner indo distributed and scallable system in the second part of the thesis.  While deploying the service in an distributed system an emphasis is taken on security of the whole architecture, therefore authentication and authorization into the system will be described in the work. Part of the thesis is dedicated to the application scalability,  with cases when servers or datacenters are down. Importance will be put on high availability of the application, both in usual day to day business and also while deploying the microservices.}
\placeForDeclarationOfAuthenticity{V~Praze}
\declarationOfAuthenticityOption{4} %volba Prohlášení (číslo 1-6)
\keywordsCS{navigace, plánování cest, multimodální plánovač, hledání nejkratších cest, otp, osrm, dijskra, a star, contracted hierarchies, distribuovaný systém, vysoká dostupnost, mikroslužba, kubernetes, eks, docker, kontejnerizace, autentikace, autorizace, api gateway, oauth2}
\keywordsEN{navigation, routing, multimodal, shortest path, otp, osrm, dijskra, a star, contracted hierarchies, distributed system, high availability, scalable service, microservice, kubernetes, eks, conteinerization, docker, authentication, authorization, api gateway, oauth2}
% \website{http://site.example/thesis} %volitelná URL práce, objeví se v tiráži - úplně odstraňte, nemáte-li URL práce

\begin{document}

% \newacronym{CVUT}{{\v C}VUT}{{\v C}esk{\' e} vysok{\' e} u{\v c}en{\' i} technick{\' e} v Praze}
% \newacronym{FIT}{FIT}{Fakulta informa{\v c}n{\' i}ch technologi{\' i}}

\begin{introduction}
	%sem napište úvod Vaší práce
V současné době je plánování cest velmi důležité a na důležitosti stále více přibývá. A s tím, jak dopravní síť začíná být čím dál více složitější a naše pohyblivost po městě začíná být čím dál více důležitější, tak také stoupá potřeba po efektivním a rychléím plánovači. Plánovač je obsažen ve většině současných chytrých mobilních telefonů a také většína leteckých či drážních společností poskytuje nějaký způsob naplánování cesty s jejich dopravními prostředky. 

Současné plánovací aplikace mají až na výjimky jedno společné omezení - to je, že plánují jen ve svém vlastním způsobu dopravy. Když využijeme aplikaci hromadné dopravy, plánovač nám ukáže pouze tramvaje, metro a jiné prostředky MHD. Podobně je to s GPS navigací, kde můžeme hledat cestu jen v silniční síti. 

Díky tomu, že současnost poskytuje ve velkých městech velké množství způsobů dopravy,  tak vidíme, že jedna z možností je tyto způsoby dopravy kombinovat. Toto je něco, což často není tolik využíváno - už jen proto, že takovýto plán obsahující více možností je složitý na složení, alespoň manuálně. Pro zkombinování více způsobů dopravy je třeba vytvořit pokročilý plánovač, který spočítá více typů dopravy. Takový plánovač se nazývá multomodální.

Tato pláce má tedy za jeden z cílů toto. Vybrat počáteční lokaci, konečnou lokaci, společně s časem odjezdu (pro tuto práci bude brán pouze případ aktuálního času)  a způsoby dopravy (ku příkladu sdílené městské prostědky, jako kola, skůtry, koloběžky, taxi, hromadná doprava) a daná aplikace vrátí seznam tras blízkých optimální trase. 

Dalším stěžejním cílem je nasazení výše zmíněné aplikace do kontejnerizovaného, distribuvaného a škálovatelného prostředí. Spouštění programů v kontejnerech se v poslední době těší veliké oblibě, většinou z důvodu potřeby vysoké dostupnosti aplikace. Takové aplikace většinou přichází s určitými potřebami - měly by  být zabezpečené a  škálovatelné. V současnosti k takovým požadavkům přichází určitá řešení. 

Jedním z řešení k nasazení takové aplikace je využití cloudových řešení, oproti využití nasazení přímo do konvenčních serverů, či virtuálních serverů.I přes to, že cloud je zajímavá alternativa, nenabízí vysokou dostupnost automaticky. Tedy pro to, aby aplikace byla vysoce dostupná, měla by být k takovým požadavkům navržena už od začátku (taková aplikace se nazývá cloud-native). Takové prostředí musí být také orchestováno, k čemu existují určité nástroje, kterým se tato práce bude také věnovat. 

Práce představí kontejner s výše zmíněnou aplikací, spolu s tím orchestrační software, díky kterému se aplikace bude spravovat. Dále představí problémy při nasazování vysoce dostupné aplikace do orchestrovaného, kontejnerizovaného prostředí. 

Struktura a cíle jsou více do detailu popsány v kapitole níže.

\end{introduction}

\chapter{Struktura a cíle práce}


Diplomová práce je rozdělena do dvou hlavních bodů. Prvním je navrhnout a popsat aplikaci či systém, který bude plánovat cesty v geoprostorových grafech. Konkrétně je omezení  na cesty ve městech, a to s použitím různých metod dopravy. Se současným rozvojem sdílených dopravních prostředků ve městech bude služba využívat jak veřejné hromadné dopravy, tak i těchto sdílených vozidel. druhou částí je návrh a popis nasazení plánovací aplikace (mikroslužby) do distribuovaného systému.

\section{Plánovač cest}


První částí je plánovač cest ve městech. Výstupem navrhovaného plánovače budou cesty s použitím jednoho typu prostředku, ale také s jejich kombinacemi (takové kombinace, které dávají smysl — tento výběr bude také v práci diskutován).  Ke správnému návrhu a pochopení problematiky hledání cest budou popsány dva základní modely - model závislý a model nezávislý na čase. Na nich budou popsány algoritmy hledající nejkratší cesty.  K následému návrhu bude použit software publikovaný pod otevřenou licencí. U tohoto software práce popíše algoritmy, dle kterých jsou cesty v grafech hledány a na nihž je software stavěn. 


\section{Nasazení mikroslužby v distribuovaném systému}



Druhou částí je navržení nasazení aplikace do distribuovaného systému Kubernetes. Bude diskutováno, proč Kubernetes byl vybrán a jeho součásti budou popsány. V nasazení má být brán důraz na několik faktorů. Jedním z ním je bezpečnost aplikace běžící v otevřeném internetu. Tedy je důležité mít komunikaci s aplikací řešenou šifrovaně. Práce bude popisovat autentikaci a autorizaci přístupů do jednotilivých částí API rozhraní aplikace. Důležitá je též vysoká dostupnost aplikace, budou tedy popsány mechanismy vysoké dostupnosti v distribuovaném systému, a to i při opakovaném nasazování. Budou diskutovány způsoby nasazení, mezi ně patřící Blue/Green deployment, Carnary releases, Rolling updates, atp. Nasazování bude řešeno automatizovaně, spolu s porovnáním poskytovatelů CD (automatizovaných deploymentů/nasazení).




\chapter{Teoretická část}
\section{Základy teorie grafů}

Pro pochopení problému hledání cest je prvně třeba definovat jednotlivé stavební bloky. Práce tedy v kapitolce níže popíše relevantní definice z teorie grafů.

\subsection{Graf}

\begin{defn}{Graf}\label{thm:graf}
	(jednoduchý neorientovaný graf) je uspořádaná dvojice $G = (V,E)$, kde V je množina vrcholů a $E$ je množina hran – množina vybraných dvouprvkových podmnožin množiny vrcholů.  [1]  
	\end{defn}

\begin{defn}{Hrana, vrchol}\label{thm:graf}

Hranu mezi vrcholy $u$ a $v$ označujeme jako $\{u, v\}$. Vrcholy spojené hranou nazýváme vrcholy sousední. Značkou $V(G)$ označujeme množinu vrcholů grafu $G$, množinu hran označujeme jako $E(G)$.
\end{defn}

\begin{defn}{Orientovaný graf}\label{thm:graf}

Orientovaný graf je uspořádaná dvojice $D = (V, E)$, kde $E \subseteq V \times V$ . 
\end{defn}

Všechny dále zmíněné grafy budou orientované, tj. orientace hrany je důležitá.

% [1] Za ́klady Teorie Graf ̊u  pro (nejen) informatiky 

\subsection{Ohodnocení hran}

Tím hlavním rozdílem mezi časově závislým a časově nezávislým grafem je právě způsob ohodnocení hran. U časově nezávislého modelu nám stačí přiřadit hraně konstatní hodnotu. U časově závislého je ohodnocení hran v ruzné denní časy různé.

\subsection{Cesta}

\begin{defn}{Cesta}\label{thm:graf}

Podgrafu $H \subseteq	 G$, který je isomorfní nějaké cestě, říkáme cesta v $G$. 
\end{defn}

Jinak řečeno, cesta $P$ je sekvence uzlů tak, že pro každý $1 \leq	i < k$ platí podmínka $(vi, vi+1) \in	 E$. 

\begin{figure}[H]\centering
	% \includegraphics[width=0.5\textwidth, angle=30]{cvut-logo-bw}
	\includesvg[width=0.5\textwidth]{graphs/path.drawio.svg}

	\caption[Příklad obrázku]{Cesta délky $n$}\label{fig:float}
\end{figure}

% \includesvg[width=\textwidth]{graphs/path.drawio.svg}

(obrázek cesty)

\begin{defn}{Kružnice}\label{thm:graf}

Kružnice délky $n$ má $n \geq 3$ vrcholů spojených do jednoho cyklu $n$ hranami.
\end{defn}

(obrázek kružnice)

\begin{defn}{Délka cesty}\label{thm:graf}

Délka cesty je součet jejich ohocnocení hran podél cesty.
\end{defn}


Plánování cest

Modely

Časově nezávislý model

Problém nejdřívějšího příjezdu


Hledání nejkratší cesty

many to many

one to many

many to one


Algoritmy k hledání nejkratší cesty


\chapter{Analýza a návrh}

\chapter{Realizace}

\begin{conclusion}
	%sem napište závěr Vaší práce
\end{conclusion}

\bibliographystyle{csn690}
\bibliography{mybibliographyfile}

\appendix

\chapter{Seznam použitých zkratek}
% \printglossaries
\begin{description}
	\item[GUI] Graphical user interface
	\item[XML] Extensible markup language
\end{description}


% % % % % % % % % % % % % % % % % % % % % % % % % % % % 
% % Tuto kapitolu z výsledné práce ODSTRAŇTE.
% % % % % % % % % % % % % % % % % % % % % % % % % % % % 
% 
% \chapter{Návod k~použití této šablony}
% 
% Tento dokument slouží jako základ pro napsání závěrečné práce na Fakultě informačních technologií ČVUT v~Praze.
% 
% \section{Výběr základu}
% 
% Vyberte si šablonu podle druhu práce (bakalářská, diplomová), jazyka (čeština, angličtina) a kódování (ASCII, \mbox{UTF-8}, \mbox{ISO-8859-2} neboli latin2 a nebo \mbox{Windows-1250}). 
% 
% V~české variantě naleznete šablony v~souborech pojmenovaných ve formátu práce\_kódování.tex. Typ může být:
% \begin{description}
% 	\item[BP] bakalářská práce,
% 	\item[DP] diplomová (magisterská) práce.
% \end{description}
% Kódování, ve kterém chcete psát, může být:
% \begin{description}
% 	\item[UTF-8] kódování Unicode,
% 	\item[ISO-8859-2] latin2,
% 	\item[Windows-1250] znaková sada 1250 Windows.
% \end{description}
% V~případě nejistoty ohledně kódování doporučujeme následující postup:
% \begin{enumerate}
% 	\item Otevřete šablony pro kódování UTF-8 v~editoru prostého textu, který chcete pro psaní práce použít -- pokud můžete texty s~diakritikou normálně přečíst, použijte tuto šablonu.
% 	\item V~opačném případě postupujte dále podle toho, jaký operační systém používáte:
% 	\begin{itemize}
% 		\item v~případě Windows použijte šablonu pro kódování \mbox{Windows-1250},
% 		\item jinak zkuste použít šablonu pro kódování \mbox{ISO-8859-2}.
% 	\end{itemize}
% \end{enumerate}
% 
% 
% V~anglické variantě jsou šablony pojmenované podle typu práce, možnosti jsou:
% \begin{description}
% 	\item[bachelors] bakalářská práce,
% 	\item[masters] diplomová (magisterská) práce.
% \end{description}
% 
% \section{Použití šablony}
% 
% Šablona je určena pro zpracování systémem \LaTeXe{}. Text je možné psát v~textovém editoru jako prostý text, lze však také využít specializovaný editor pro \LaTeX{}, např. Kile.
% 
% Pro získání tisknutelného výstupu z~takto vytvořeného souboru použijte příkaz \verb|pdflatex|, kterému předáte cestu k~souboru jako parametr. Vhodný editor pro \LaTeX{} toto udělá za Vás. \verb|pdfcslatex| ani \verb|cslatex| \emph{nebudou} s~těmito šablonami fungovat.
% 
% Více informací o~použití systému \LaTeX{} najdete např. v~\cite{wikilatex}.
% 
% \subsection{Typografie}
% 
% Při psaní dodržujte typografické konvence zvoleného jazyka. České \uv{uvozovky} zapisujte použitím příkazu \verb|\uv|, kterému v~parametru předáte text, jenž má být v~uvozovkách. Anglické otevírací uvozovky se v~\LaTeX{}u zadávají jako dva zpětné apostrofy, uzavírací uvozovky jako dva apostrofy. Často chybně uváděný symbol "{} (palce) nemá s~uvozovkami nic společného.
% 
% Dále je třeba zabránit zalomení řádky mezi některými slovy, v~češtině např. za jednopísmennými předložkami a spojkami (vyjma \uv{a}). To docílíte vložením pružné nezalomitelné mezery -- znakem \texttt{\textasciitilde}. V~tomto případě to není třeba dělat ručně, lze použít program \verb|vlna|.
% 
% Více o~typografii viz \cite{kobltypo}.
% 
% \subsection{Obrázky}
% 
% Pro umožnění vkládání obrázků je vhodné použít balíček \verb|graphicx|, samotné vložení se provede příkazem \verb|\includegraphics|. Takto je možné vkládat obrázky ve formátu PDF, PNG a JPEG jestliže používáte pdf\LaTeX{} nebo ve formátu EPS jestliže používáte \LaTeX{}. Doporučujeme preferovat vektorové obrázky před rastrovými (vyjma fotografií).
% 
% \subsubsection{Získání vhodného formátu}
% 
% Pro získání vektorových formátů PDF nebo EPS z~jiných lze použít některý z~vektorových grafických editorů. Pro převod rastrového obrázku na vektorový lze použít rasterizaci, kterou mnohé editory zvládají (např. Inkscape). Pro konverze lze použít též nástroje pro dávkové zpracování běžně dodávané s~\LaTeX{}em, např. \verb|epstopdf|.
% 
% \subsubsection{Plovoucí prostředí}
% 
% Příkazem \verb|\includegraphics| lze obrázky vkládat přímo, doporučujeme však použít plovoucí prostředí, konkrétně \verb|figure|. Například obrázek \ref{fig:float} byl vložen tímto způsobem. Vůbec přitom nevadí, když je obrázek umístěn jinde, než bylo původně zamýšleno -- je tomu tak hlavně kvůli dodržení typografických konvencí. Namísto vynucování konkrétní pozice obrázku doporučujeme používat odkazování z~textu (dvojice příkazů \verb|\label| a \verb|\ref|).
% 
% \begin{figure}\centering
% 	\includegraphics[width=0.5\textwidth, angle=30]{cvut-logo-bw}
% 	\caption[Příklad obrázku]{Ukázkový obrázek v~plovoucím prostředí}\label{fig:float}
% \end{figure}
% 
% \subsubsection{Verze obrázků}
% 
% % Gnuplot BW i barevně
% Může se hodit mít více verzí stejného obrázku, např. pro barevný či černobílý tisk a nebo pro prezentaci. S~pomocí některých nástrojů na generování grafiky je to snadné.
% 
% Máte-li například graf vytvořený v programu Gnuplot, můžete jeho černobílou variantu (viz obr. \ref{fig:gnuplot-bw}) vytvořit parametrem \verb|monochrome dashed| příkazu \verb|set term|. Barevnou variantu (viz obr. \ref{fig:gnuplot-col}) vhodnou na prezentace lze vytvořit parametrem \verb|colour solid|.
% 
% \begin{figure}\centering
% 	\includegraphics{gnuplot-bw}
% 	\caption{Černobílá varianta obrázku generovaného programem Gnuplot}\label{fig:gnuplot-bw}
% \end{figure}
% 
% \begin{figure}\centering
% 	\includegraphics{gnuplot-col}
% 	\caption{Barevná varianta obrázku generovaného programem Gnuplot}\label{fig:gnuplot-col}
% \end{figure}
% 
% 
% \subsection{Tabulky}
% 
% Tabulky lze zadávat různě, např. v~prostředí \verb|tabular|, avšak pro jejich vkládání platí to samé, co pro obrázky -- použijte plovoucí prostředí, v~tomto případě \verb|table|. Například tabulka \ref{tab:matematika} byla vložena tímto způsobem.
% 
% \begin{table}\centering
% 	\caption[Příklad tabulky]{Zadávání matematiky}\label{tab:matematika}
% 	\begin{tabular}{|l|l|c|c|}\hline
% 		Typ		& Prostředí		& \LaTeX{}ovská zkratka	& \TeX{}ovská zkratka	\tabularnewline \hline \hline
% 		Text		& \verb|math|		& \verb|\(...\)|	& \verb|$...$|		\tabularnewline \hline
% 		Displayed	& \verb|displaymath|	& \verb|\[...\]|	& \verb|$$...$$|	\tabularnewline \hline
% 	\end{tabular}
% \end{table}
% 
% % % % % % % % % % % % % % % % % % % % % % % % % % % % 

\chapter{Obsah přiloženého CD}

%upravte podle skutecnosti

\begin{figure}
	\dirtree{%
		.1 readme.txt\DTcomment{stručný popis obsahu CD}.
		.1 exe\DTcomment{adresář se spustitelnou formou implementace}.
		.1 src.
		.2 impl\DTcomment{zdrojové kódy implementace}.
		.2 thesis\DTcomment{zdrojová forma práce ve formátu \LaTeX{}}.
		.1 text\DTcomment{text práce}.
		.2 thesis.pdf\DTcomment{text práce ve formátu PDF}.
		.2 thesis.ps\DTcomment{text práce ve formátu PS}.
	}
\end{figure}

\end{document}
